%%%%%%%%%%%%%%%%%%%%%%%%%%%%%%%%%%%%%%%%%%%%%%%%%%%%%%%%%%%%%%%
%
% Welcome to Overleaf --- just edit your LaTeX on the left,
% and we'll compile it for you on the right. If you open the
% 'Share' menu, you can invite other users to edit at the same
% time. See www.overleaf.com/learn for more info. Enjoy!
%
%%%%%%%%%%%%%%%%%%%%%%%%%%%%%%%%%%%%%%%%%%%%%%%%%%%%%%%%%%%%%%%

% Inbuilt themes in beamer
\documentclass{beamer}

%packages:
% \usepackage{tfrupee}
% \usepackage{amsmath}
% \usepackage{amssymb}
% \usepackage{gensymb}
% \usepackage{txfonts}

% \def\inputGnumericTable{}

% \usepackage[latin1]{inputenc}                                 
% \usepackage{color}                                            
% \usepackage{array}                                            
% \usepackage{longtable}                                        
% \usepackage{calc}                                             
% \usepackage{multirow}                                         
% \usepackage{hhline}                                           
% \usepackage{ifthen}
% \usepackage{caption} 
% \captionsetup[table]{skip=3pt}  
% \providecommand{\pr}[1]{\ensuremath{\Pr\left(#1\right)}}
% \providecommand{\cbrak}[1]{\ensuremath{\left\{#1\right\}}}
% %\renewcommand{\thefigure}{\arabic{table}}
% \renewcommand{\thetable}{\arabic{table}}      

\setbeamertemplate{caption}[numbered]{}

\usepackage{enumitem}
\usepackage{tfrupee}
\usepackage{amsmath}
\usepackage{amssymb}
\usepackage{gensymb}
\usepackage{graphicx}
\usepackage{txfonts}

\def\inputGnumericTable{}

\usepackage[latin1]{inputenc}                                 
\usepackage{color}                                  \usepackage{textcomp, gensymb}         
\usepackage{array}                                            
\usepackage{longtable}                                        
\usepackage{calc}                                             
\usepackage{multirow}                                         
\usepackage{hhline}                                           
\usepackage{ifthen}
\usepackage{caption} 
\providecommand{\pr}[1]{\ensuremath{\Pr\left(#1\right)}}
\providecommand{\cbrak}[1]{\ensuremath{\left\{#1\right\}}}
\renewcommand{\thefigure}{\arabic{table}}
\renewcommand{\thetable}{\arabic{table}}   
\providecommand{\brak}[1]{\ensuremath{\left(#1\right)}}

% Theme choice:
\usetheme{CambridgeUS}

% Title page details: 
\title{Assignment 8} 
\author[CS21BTECH11017]{G HARSHA VARDHAN REDDY (CS21BTECH11017)}
\date{\today}
\logo{\large{AI1110}}


\begin{document}

% Title page frame
\begin{frame}
    \titlepage 
\end{frame}
\logo{}


% Outline frame
\begin{frame}{Outline}
    \tableofcontents
\end{frame}

\section{Abstract}
\begin{frame}{Abstract}
\begin{block}{} This document contains $16^{th}$ problem from the chapter 4 in the book \textbf{Papoulis Pillai Probability RandomVariables and Stochastic Processes.}
\end{block}
    
\end{frame}

\section{Problem Statement}
\begin{frame}{Problem Statement}
    \begin{block} {Problem} Show that if $X(\zeta)\leq X(\zeta) $ for every $\zeta \in S$ then $F_X(\omega) \ge F_Y(\omega)$ for every $\omega$.
    \end{block}
\end{frame}
\begin{frame}{Random Variables}
We are given an experiment specified by the space $S$, the field of subsets of $S$
called events, and the probability assigned to these events. To every outcome $\zeta$ of this
experiment, we assign a number $X(\zeta)$ . We have thus created a function $X$ with domain
the set $S$ and range a set of numbers. This function is called random variable
\end{frame}
\begin{frame}{Cumulative Distribution Function}
The cumulative distribution function for a random variable $X$  $F_X(\omega)$ can be defined as below:

\begin{block}{}
\begin{align}
         F_X(\omega)&= \pr{X \leq \omega} \text{ for every } \omega \in S
         \label{eq1}
\end{align}
\end{block}

\end{frame}


\section{Solution}
\begin{frame}{Solution}
Given
\begin{align}
    X(\zeta)&\leq Y(\zeta) \text{    }\forall \zeta \in S
    \label{eq2}
\end{align}
Let's say 
\begin{align}
    Y(\zeta_i)&\leq \omega 
\end{align}
then, From \eqref{eq2}
\begin{align}
    X(\zeta_i)&\leq \omega 
\end{align}
Hence,\\
\begin{align}
    \cbrak {Y \leq \omega } \subseteq \cbrak{X\leq \omega}
    \label{eq5}
\end{align}
And we know that 
\begin{align}
    A \subseteq B \implies P(A) \leq P(B) \label{eq6}
\end{align}
\end{frame}
\begin{frame}{}
Therefore from \eqref{eq5},\eqref{eq6}
\begin{align}
    P(Y \leq \omega) \leq P(X \leq \omega)
\end{align}
And From \eqref{eq1}
\begin{align}
    P(Y \leq \omega) \leq P(X \leq \omega) \implies F_X(\omega) \leq F_Y(\omega)
    \label{eq8}
\end{align}
If $Y(\zeta_i) > \omega$  then
\begin{align}
    \cbrak{Y \leq \omega} = \Phi
    \implies P(Y \leq \omega) = F_Y(\omega)=0
\end{align}
And as $P(x)\ge 0$
\begin{align}
    F_X(\omega) &\ge 0\\
    \implies F_X(\omega) &\ge F_Y(\omega) \label{eq11}
\end{align}
From \eqref{eq8} and \eqref{eq11} it is clear that
\begin{align}
    X(\zeta)&\leq Y(\zeta) \implies F_X(\omega) \ge F_Y(\omega)
\end{align}
\end{frame}

\end{document}